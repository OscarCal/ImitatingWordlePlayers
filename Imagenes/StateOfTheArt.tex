Pese a esto, su similitud con otros juegos de puzzles y su gran popularidad han hecho posible la existencia de varias fuentes
que podremos utilizar como base de nuestro proyecto.

\section{Mastermind}

Uno de los puzzles que inspiraron la creación de Wordle es Mastermind, este juego se basa en intentar averiguar un código oculto compuesto una secuencia de cuatro
fichas de seis posibles colores donde el orden importa y se permiten repetir los colores. Este código es propuesto por el "codificador" (uno de los dos jugadores) y la
labor del "decodificador" es intentar averiguar este código mediante propuestas que se van refinando gracias a las pistas que el codificador ofrece tras cada intento. 
Las pistas que se facilitan al "decodificador" se rigen por las siguientes normas.
\begin{enumerate}
  \item Se devuelve una ficha de color negro por cada ficha de la combinación introducida que coincida en posición y color con la combinación oculta.
  \item Se devuelve una ficha blanca por cada ficha de la combinación introducida cuyo color aparezca en la combinación oculta pero no se encuentre en la posición correcta.
  \item Si la combinación introducida contiene fichas del mismo color solo se otorgarán tantas fichas como fichas de ese color aparezcan en la combinación oculta.
  Las fichas sobrantes en la combinación introducida no pueden provocar la entrega de ninguna ficha.
\end{enumerate}

El "decodificador" tiene hasta doce intentos para averiguar la combinación, el juego termina cuando el "codificador" entrega cuatro fichas negras o cuando se alcanza el límite 
de intentos. Podemos observar claras similitudes entre Wordle y Mastermind, especialmente a nivel de jugabilidad ya que ambos juegos se basan en adivinar una combinación de caracteres
ocultos y en cada intento se reciben pistas. Además, el formato de las pistas es prácticamente idéntico salvo por el formato de los colores y que la ausencia de ficha en Mastermind se 
traduce en devolver un cuadrado gris en Wordle. Pese a todo esto, creemos que es importante presentar las diferencias más relevantes entre cada juego:

\begin{enumerate}
  \item En Mastermind solo se indica el número de fichas en la posición correcta, en Wordle se indica además cuales son.
  \item En Mastermind las códigos continen seis colores distintos y poseen 4 posiciones, en Wordle los códigos contienen veintiséis letras distintas y poseen 5 posiciones.
  \item En Mastermind todas las combinaciones de colores son validas tanto como código objetivo como código introducido, en Wordle solo se permite utilizar palabras
  del idioma inglés pertenecientes a una lista fijada por el desarrollador.
  \item Mastermind permite hasta doce intentos, Wordle permite únicamente seis.
\end{enumerate}